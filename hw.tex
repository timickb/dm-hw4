\documentclass[a4paper,11pt]{article}

\usepackage[utf8]{inputenc}
\usepackage[english,russian]{babel}
\usepackage{titling}
\usepackage{titlesec}
\usepackage{amsfonts,amsmath,amssymb,amsthm,mathtools}
\usepackage{icomma}
\usepackage{cmap}
\usepackage{mathtext}

\titleformat{\section}
  {\normalfont\normalsize}{\fbox{\textbf{\thesection}}}{1em}{}

\setlength{\parindent}{0em}
\setlength{\droptitle}{-7em}

\usepackage[left=2cm,right=2cm,
    top=2cm,bottom=2cm,bindingoffset=0cm]{geometry}

\title{Домашнее задание 4}
\author{Батрутдинов Тимур, БПИ206}

\begin{document}
    \maketitle

    \section{$P \subseteq \mathbb{R} \times \mathbb{R}$}

    \section{Пусть $R \subseteq A \times B$ функционально.}

    \section{$f: A \implies B$ и $g: A \rightarrow B$. Доказать, что $f \cup g: A \rightarrow B \Leftrightarrow f = g$}

    \section{$f: A \implies B$ и $g: B \rightarrow C$. Доказать следствие: $g \circ f$ - инъекция $\implies f$ - тоже инъекция.}

    \section{Доказать: $f: A \rightarrow B$ инъективна $\Leftrightarrow \forall{C} \forall{g, h: C \rightarrow A}: (f \circ g = f\circ h \implies g = h)$}

    \section{}
    a). $\mathbb{Q}^{\underline{3}}$

    \vspace{\baselineskip}
    b).$\mathbb{R}^{\mathbb{Q}}$

    \vspace{\baselineskip}
    c). $\mathbb{R}^{\mathbb{R}\times \mathbb{Z}}$

    \section{Пусть $A \cap B = \varnothing$. Доказать: $C^{A\cup B} \sim C^{A} \times C^{B}$}

    \section{$P_1(A)$ - множество всех подмножеств множества $A$ вида $\{x\}$. Доказать: $P_1(A) \sim A$ $\forall A$}

    \section{Доказать, что для любых $A, B, C$ верно, используя характ. функции:}
    a). $(A \backslash B) \backslash C = (A \backslash C) \cup (B \backslash C)$

    \vspace{\baselineskip}
    b). $(A \backslash B) \cup B = A \Leftrightarrow B \subseteq A$

    \section{Доказать с помощью теоремы Кантора-Бернштейна-Шрёдера:}
    a). $\mathbb{N}^{\mathbb{N}\times\mathbb{Q}}\times \mathbb{N} \sim \mathbb{R}^{\mathbb{Q}}$

    \vspace{\baselineskip}
    b). $\underline{5}^{\mathbb{N}}\sim \underline{3}^{\mathbb{N}}$

    \vspace{\baselineskip}
    c). любой квадрат (с внутренностью) и любой круг на плоскости равномощны друг другу; (подумайте
    о движениях и других геометрических преобразованиях плоскости)

    \vspace{\baselineskip}
    d). множество всевозможных треугольников на плоскости равномощно $\mathbb{R}$.



    

\end{document}