\documentclass[a4paper,11pt]{article}

\usepackage[utf8]{inputenc}
\usepackage[english,russian]{babel}
\usepackage{titling}
\usepackage{titlesec}
\usepackage{amsfonts,amsmath,amssymb,amsthm,mathtools}
\usepackage{icomma}
\usepackage{cmap}
\usepackage{mathtext}

\titleformat{\section}
  {\normalfont\normalsize}{\fbox{\textbf{\thesection}}}{1em}{}

\setlength{\parindent}{0em}
\setlength{\droptitle}{-7em}

\usepackage[left=2cm,right=2cm,
    top=2cm,bottom=2cm,bindingoffset=0cm]{geometry}

\title{Домашнее задание 4}
\author{Батрутдинов Тимур, БПИ206}

\begin{document}
    \maketitle

    \section{$P \subseteq \mathbb{R} \times \mathbb{R}$}

    a). $\left\{\begin{aligned}
        P(x) = x, x\in\mathbb{R}\backslash\{0\}\\
        P(1) = 0
    \end{aligned}\right.$

    \vspace{\baselineskip}
    b). $\left\{\begin{aligned}
      P(x) = x, x\in\mathbb{R}\backslash\{0\}\\
      P(0) = 1
  \end{aligned}\right.$

    \section{Пусть $R \subseteq A \times B$ функционально.}

    \section{$f: A \rightarrow B$ и $g: A \rightarrow B$. Доказать, что $f \cup g: A \rightarrow B \Leftrightarrow f = g$}

    \section{$f: A \rightarrow B$ и $g: B \rightarrow C$. Доказать следствие: $g \circ f$ - инъекция $\implies f$ - тоже инъекция.}

    \section{Доказать: $f: A \rightarrow B$ инъективна $\Leftrightarrow \forall{C} \forall{g, h: C \rightarrow A}: (f \circ g = f\circ h \implies g = h)$}

    \section{}
    a). $\mathbb{Q}^{\underline{3}}$: $\{(0, \frac{1}{2}), (1, \frac{2}{3}), (2, \frac{4}{7})\}$

    \vspace{\baselineskip}
    b).$\mathbb{R}^{\mathbb{Q}}$: $\{(\frac{1}{2}, \sqrt{2}), (\frac{2}{3}, \pi), (\frac{3}{4}, \sqrt{5})\}$

    \vspace{\baselineskip}
    c). $\mathbb{R}^{\mathbb{R}\times \mathbb{Z}}$:
        $\{((4.13,8), 17.6), ((0.21,2093), 0.2), ((3.71,3), 9)\}$

    \section{Пусть $A \cap B = \varnothing$. Доказать: $C^{A\cup B} \sim C^{A} \times C^{B}$}
    1) $|A| + |B| = |A \cup B| - |A \cap B| = |A\cup B| - 0 = |A\cup B|$.

    Пусть $|A| = n, |B| = m, |C| = k$

    \vspace{\baselineskip}
    Известно, что $|A^{B}| = |A|^{|B|}$.

    Также очевиден факт, что $|C^A\times C^B| = |C^A|\cdot |C^B|$

    \vspace{\baselineskip}
    Тогда $|C^{A\cup B}| = k^{n+m}, |C^A| = k^n, |C^B| = k^m$,

    $|C^A\times C^B| = |C^A|\cdot |C^B| = k^n\cdot k^m = k^{n+m}$.

    Выходит, что $|C^{A\cup B}| = |C^A\times C^B| = k^{n+m}$, а значит, $C^{A\cup B} \sim C^{A} \times C^{B}$.

    \section{$P_1(A)$ - множество всех подмножеств множества $A$ вида $\{x\}$. Доказать: $P_1(A) \sim A$ $\forall A$}

    Пусть $A = \{a_0, a_1, a_2\dots\}$. Тогда $P_1(A) = \{\{a_0\},\{a_1\},\{a_2\}\dots\}$

    \vspace{\baselineskip}
    1) Пусть А - конечное и $|A| = n$. Тогда есть $n$ способов выбрать первый синглтон в $P_1(A)$,
    $n-1$ способ выбрать 2-й синглтон, $n-2$ - 3-й, и так далее. $n$-й синглтон можно выбрать
    только одним способом, а значит, других синглтонов в $P_1(A)$ быть не может и их ровно $n$,
    т.е $|P_1(A)| = |A|$.

    \vspace{\baselineskip}
    2) В общем случае каждому элементу из $A$ соответствует ровно один синглтон из $P_1(A)$,
    потому что для любой сущности есть только один способ создать одноэлементное множество,
    состоящее из него одного.

    \section{Доказать, что для любых $A, B, C$ верно, используя характ. функции:}
    a). $(A \backslash B) \backslash C = (A \backslash C) \cup (B \backslash C)$

    \vspace{\baselineskip}
    b). $(A \backslash B) \cup B = A \Leftrightarrow B \subseteq A$

    \section{Доказать с помощью теоремы Кантора-Бернштейна-Шрёдера:}
    a). $\mathbb{N}^{\mathbb{N}\times\mathbb{Q}}\times \mathbb{N} \sim \mathbb{R}^{\mathbb{Q}}$

    \vspace{\baselineskip}
    b). $\underline{5}^{\mathbb{N}}\sim \underline{3}^{\mathbb{N}}$

    $$\underline{5}^{\mathbb{N}} = \left\{ f: \mathbb{N}\rightarrow \{0,1,2,3,4\} \right\}$$
    $$\underline{3}^{\mathbb{N}} = \left\{ f: \mathbb{N}\rightarrow \{0,1,2\} \right\}$$

    $\underline{5}^{\mathbb{N}} \sim \mathbb{N}$, так как каждому натуральному числу можно поставить в соответствие
    \textbf{ровно один} его остаток от деления на 5,

    $\underline{3}^{\mathbb{N}} \sim \mathbb{N}$, так как каждому натуральному числу можно поставить в соответствие
    \textbf{ровно один} его остаток от деления на 3.

    Оба множества равномощны натуральному ряду, значит, они равномощны друг другу.

    \vspace{\baselineskip}
    c). любой квадрат (с внутренностью) и любой круг на плоскости равномощны друг другу; (подумайте
    о движениях и других геометрических преобразованиях плоскости)

    \vspace{\baselineskip}
    Обозначим множество квадратов на плоскости как 
    $S = \{ (a, b)\in\mathbb{R} \vert a\cdot b$ -  площадь квадрата$\}$;

    Обозначим множество кругов на плоскости как
    $S = \{ (x, y, r)\in\mathbb{R} \vert x, y$ -  координаты центра, $r$ - радиус$\}$.

    \vspace{\baselineskip}
    d). множество всевозможных треугольников на плоскости равномощно $\mathbb{R}$.



    

\end{document}
